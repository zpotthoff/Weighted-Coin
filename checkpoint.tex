\documentclass{article}
\usepackage{graphicx} % Required for inserting images
\usepackage{amssymb}
\usepackage{amsmath}
\usepackage[utf8]{inputenc}

\title{CS91T Spring 2023\\Final Project Checkpoint}
\author{Austin Burgess, Alex Rimerman, Zach Potthoff}
\date{April 17, 2023}


\begin{document}

\maketitle

\section{Project Goal}

Our main goal for our project would be to create a Bogo Coin in Python, that
produces heads with probability p, and tails with probability $1 - p$, given 
a random p from the user. We will do this by using a function for a fair coin
and manipulating the results in order to gain the correct probability.

\section{Methods Used and/or Related Work}

We plan on doing an implementation project. For our project, within our 
code we will utilize a fair coin algorithm (this will most likely be
through Python's built in random library). Additionally, before we begin 
coding, we will do some reading on Markov Chain Monte Carlos. As we have 
 already seen, the Metropolis-Hastings algorithm seems that it will 
 be important and relevant to our project. At this moment, 
we do not really know what this is and have not heard of these prior 
to this, which is why we are reading about them. We do not have much more 
for this section as we are trying to create the randomized algorithm 
ourselves.

\section{Staged Development Plan}

\begin{enumerate}
    \item Understand Markov Chain Monte Carlo
    \begin{enumerate}
        \item Metropolis-Hastings algorithm
    \end{enumerate}
    \item Set up the fair coin function
    \item Get the bogo coin function to work when $p = .25$
    \item Get the bogo coin function to work with a certain error probability
    for all hundredths numbers
    \item Get the bogo coin function to work with a certain error probability
    for all thousandth numbers
    \item Get the bogo coin function to work with a certain error probability
    for all rational numbers (semi-stretch, tbd)
    \item Get the bogo coin function to work with a certain error probability
    for all numbers (Stretch)
\end{enumerate} 

\section{Measure of Success}

We will measure the success by actually flipping the coins and seeing how 
close to the actual probability we get. We are thinking about plotting this on
a graph and see what the distribution looks like. Most importantly will be the
comparison of how accurate our function behaves compared to the actual $p$ value.


\section{Plans for Analyzing Results}

We'll experiment with a number of different $p$ values, starting with simple ones such as $p = 0.25$. 
We can then compare this percentage with the number of heads our algorithm produces, validating (or invalidating) 
our results. We also will use the probabilistic method to ensure a desired result given some bound on error. 

\end{document}
